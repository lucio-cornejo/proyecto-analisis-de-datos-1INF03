% Options for packages loaded elsewhere
\PassOptionsToPackage{unicode}{hyperref}
\PassOptionsToPackage{hyphens}{url}
\PassOptionsToPackage{dvipsnames,svgnames,x11names}{xcolor}
%
\documentclass[
  letterpaper,
  DIV=11,
  numbers=noendperiod]{scrartcl}
\usepackage{amsmath,amssymb}
\usepackage{lmodern}
\usepackage{iftex}
\ifPDFTeX
  \usepackage[T1]{fontenc}
  \usepackage[utf8]{inputenc}
  \usepackage{textcomp} % provide euro and other symbols
\else % if luatex or xetex
  \usepackage{unicode-math}
  \defaultfontfeatures{Scale=MatchLowercase}
  \defaultfontfeatures[\rmfamily]{Ligatures=TeX,Scale=1}
\fi
% Use upquote if available, for straight quotes in verbatim environments
\IfFileExists{upquote.sty}{\usepackage{upquote}}{}
\IfFileExists{microtype.sty}{% use microtype if available
  \usepackage[]{microtype}
  \UseMicrotypeSet[protrusion]{basicmath} % disable protrusion for tt fonts
}{}
\makeatletter
\@ifundefined{KOMAClassName}{% if non-KOMA class
  \IfFileExists{parskip.sty}{%
    \usepackage{parskip}
  }{% else
    \setlength{\parindent}{0pt}
    \setlength{\parskip}{6pt plus 2pt minus 1pt}}
}{% if KOMA class
  \KOMAoptions{parskip=half}}
\makeatother
\usepackage{xcolor}
\IfFileExists{xurl.sty}{\usepackage{xurl}}{} % add URL line breaks if available
\IfFileExists{bookmark.sty}{\usepackage{bookmark}}{\usepackage{hyperref}}
\hypersetup{
  pdftitle={Informe parcial},
  colorlinks=true,
  linkcolor={blue},
  filecolor={Maroon},
  citecolor={Blue},
  urlcolor={Blue},
  pdfcreator={LaTeX via pandoc}}
\urlstyle{same} % disable monospaced font for URLs
\usepackage{longtable,booktabs,array}
\usepackage{calc} % for calculating minipage widths
% Correct order of tables after \paragraph or \subparagraph
\usepackage{etoolbox}
\makeatletter
\patchcmd\longtable{\par}{\if@noskipsec\mbox{}\fi\par}{}{}
\makeatother
% Allow footnotes in longtable head/foot
\IfFileExists{footnotehyper.sty}{\usepackage{footnotehyper}}{\usepackage{footnote}}
\makesavenoteenv{longtable}
\usepackage{graphicx}
\makeatletter
\def\maxwidth{\ifdim\Gin@nat@width>\linewidth\linewidth\else\Gin@nat@width\fi}
\def\maxheight{\ifdim\Gin@nat@height>\textheight\textheight\else\Gin@nat@height\fi}
\makeatother
% Scale images if necessary, so that they will not overflow the page
% margins by default, and it is still possible to overwrite the defaults
% using explicit options in \includegraphics[width, height, ...]{}
\setkeys{Gin}{width=\maxwidth,height=\maxheight,keepaspectratio}
% Set default figure placement to htbp
\makeatletter
\def\fps@figure{htbp}
\makeatother
\setlength{\emergencystretch}{3em} % prevent overfull lines
\providecommand{\tightlist}{%
  \setlength{\itemsep}{0pt}\setlength{\parskip}{0pt}}
\setcounter{secnumdepth}{-\maxdimen} % remove section numbering
% Make \paragraph and \subparagraph free-standing
\ifx\paragraph\undefined\else
  \let\oldparagraph\paragraph
  \renewcommand{\paragraph}[1]{\oldparagraph{#1}\mbox{}}
\fi
\ifx\subparagraph\undefined\else
  \let\oldsubparagraph\subparagraph
  \renewcommand{\subparagraph}[1]{\oldsubparagraph{#1}\mbox{}}
\fi
\KOMAoption{captions}{tableheading}
\makeatletter
\makeatother
\makeatletter
\@ifpackageloaded{caption}{}{\usepackage{caption}}
\AtBeginDocument{%
\renewcommand*\contentsname{Table of contents}
\renewcommand*\listfigurename{List of Figures}
\renewcommand*\listtablename{List of Tables}
\renewcommand*\figurename{Figure}
\renewcommand*\tablename{Table}
}
\@ifpackageloaded{float}{}{\usepackage{float}}
\floatstyle{ruled}
\@ifundefined{c@chapter}{\newfloat{codelisting}{h}{lop}}{\newfloat{codelisting}{h}{lop}[chapter]}
\floatname{codelisting}{Listing}
\newcommand*\listoflistings{\listof{codelisting}{List of Listings}}
\makeatother
\makeatletter
\@ifpackageloaded{caption}{}{\usepackage{caption}}
\@ifpackageloaded{subcaption}{}{\usepackage{subcaption}}
\makeatother
\makeatletter
\makeatother
\ifLuaTeX
  \usepackage{selnolig}  % disable illegal ligatures
\fi

\title{Informe parcial}
\author{}
\date{2022-06-10}

\begin{document}
\maketitle

\renewcommand*\contentsname{Table of contents}
{
\hypersetup{linkcolor=}
\setcounter{tocdepth}{4}
\tableofcontents
}
\hypertarget{caruxe1tula}{%
\subsection{CARÁTULA}\label{caruxe1tula}}

\begin{itemize}
\item
  \textbf{Curso:} Análisis de Datos
\item
  \textbf{Profesor:} Stefany Neciosup
\item
  \textbf{Código del curso:} 1INF03
\item
  \textbf{Fecha de Entrega:} 11/06/2022
\item
  \textbf{Integrantes:}

  \begin{itemize}
  \tightlist
  \item
    Richle Gianotti, Renzo Ernesto - 20180368
  \item
    Cornejo Ramírez, Lucio Enrique - 20192058
  \item
    Vivas Alejandro, Claudia Mirela - 20141150
  \item
    Mejia Padilla, Andrea Adela - 20180824
  \end{itemize}
\end{itemize}

\hypertarget{introducciuxf3n}{%
\subsection{INTRODUCCIÓN}\label{introducciuxf3n}}

Para entender la dinámica de la industria musical, antes de nada, es
necesario saber que no se trata de una sola, sino de varias, diferentes,
estrechamente relacionadas entre sí, pero que parten de lógicas y
estructuras distintas. La industria musical en su conjunto vive de la
creación y la explotación de la propiedad intelectual musical.
Compositores y letristas crean canciones, letras y arreglos que se
interpretan en directo sobre el escenario, se graban y distribuyen a los
consumidores. Esta estructura básica ha dado lugar a tres industrias
musicales centrales: la discográfica, centrada en la grabación de música
y su distribución a los consumidores; la de las licencias musicales, que
sobre todo concede licencias a empresas para la explotación de
composiciones y arreglos, y la música en vivo, centrada en producir y
promocionar espectáculos en directo, como conciertos, giras, etcétera.

En la actualidad se vienen realizando muchos estudios aplicando el campo
de Data Analytics para poder obtener resultados en muchas industrias, y
la musical no es ajena a esta moda. Es por esto que decidimos
desarrollar el presente trabajo para así descubrir las características
fuertemente asociadas a las canciones más populares en la aplicación de
música Spotify y poder usar estos rasgos, que están conectados con los
estados de ánimo, fechas de lanzamiento, nombre de las canciones, entre
otros; para analizar el por qué de esta popularidad y poder simular que
una empresa de la industria musical quiera emplear un modelo analítico
que le permita saber el índice de popularidad de una canción antes de
que esta sea soltada al mercado.

Entre los puntos que queremos saber:

\begin{itemize}
\tightlist
\item
  ¿Cuáles son los tracks más populares en Spotify?
\item
  ¿Qué características en común tienen las canciones más populares de
  Spotify?
\item
  ¿Existe una correlación entre la popularidad y alguna característica
  de las canciones?
\item
  ¿Cuánto debe durar un track según los estándares de la actualidad?
\item
  ¿Cuál es la correlación entre diferentes tracks que son populares?
\end{itemize}

\hypertarget{dataset}{%
\subsubsection{Dataset}\label{dataset}}

El dataset elegido se obtuvo utilizando el servicio API de Spotify, el
cual nos permite obtener la información de las canciones existentes en
tal plataforma. El conjunto de datos que estamos empleando consta
inicialmente de 20 columnas, las cuales representan variables como el
nombre de la canción, su popularidad, duración, el artista, la fecha de
lanzamiento, bailabilidad de la canción, su energía, volumen sonoro,
etc.

\hypertarget{capuxedtulo-i-comprensiuxf3n-del-negocio-estudio}{%
\subsection{CAPÍTULO I: COMPRENSIÓN DEL NEGOCIO /
ESTUDIO}\label{capuxedtulo-i-comprensiuxf3n-del-negocio-estudio}}

\hypertarget{descripciuxf3n-del-problema}{%
\subsubsection{Descripción del
problema}\label{descripciuxf3n-del-problema}}

En la actualidad se sabe qué canciones son las más populares en general
observando la cantidad de reproducciones que tienen cada canción, sin
embargo la industria de la música es un negocio y como tal una empresa
discográfica siempre busca que su canción sea popular para que esta
genere ingresos a la empresa. Si pudiéramos saber que características
hacen popular a una canción entonces se podrían usar a favor para
generar canciones que sean del agrado del público oyente y como
consecuencia pueda otorgar un mayor beneficio económico.

\hypertarget{objetivo-del-negocioestudio}{%
\subsubsection{Objetivo del
negocio/estudio}\label{objetivo-del-negocioestudio}}

Nuestro objetivo principal es el de indagar, descubrir y utilizar las
características que definen la popularidad de las canciones presentes en
nuestro dataset.

\hypertarget{variable-objetivo-para-el-negocio}{%
\subsubsection{Variable objetivo para el
negocio}\label{variable-objetivo-para-el-negocio}}

\begin{itemize}
\tightlist
\item
  Las variables que consideramos podrían ser más relevantes para el
  negocio son:

  \begin{itemize}
  \tightlist
  \item
    popularity (Popularidad)
  \item
    duration\_ms (La duración del track en milisegundos)
  \item
    danceability (La capacidad de baile del track)
  \end{itemize}
\end{itemize}

\hypertarget{capuxedtulo-ii-fuentes-de-informaciuxf3n}{%
\subsection{CAPÍTULO II: FUENTES DE
INFORMACIÓN}\label{capuxedtulo-ii-fuentes-de-informaciuxf3n}}

\hypertarget{origen-de-los-datos-del-proyecto}{%
\subsubsection{Origen de los datos del
proyecto}\label{origen-de-los-datos-del-proyecto}}

El dataset que empleamos en nuestro trabajo lo descargamos de
\href{https://www.kaggle.com/datasets/yamaerenay/spotify-dataset-19212020-600k-tracks?resource=download}{este}
sitio web \ldots{} precisamente, el archivo \textbf{tracks.csv}.

Como se describe en el sitio web del link previo, el dataset
\textbf{tracks.csv} se obtuvo vía el API oficial de \textbf{Spotify}.
Tal API permite descargar, para cualquier canción en Spotify,
características sobre aquella canción, tales como su duración en
milisegundos, volumen, qué tan bailable es, etc.

Así, cada fila en \textbf{tracks.csv} representa a una canción diferente
en Spotify, y, las columnas del dataset representan variables asociadas
a cada canción.

\hypertarget{descripciuxf3n-del-universo-y-muestra}{%
\subsubsection{Descripción del universo y
muestra}\label{descripciuxf3n-del-universo-y-muestra}}

En nuestro caso, el \textbf{universo} consistiría el el conjunto de
canciones de Spotify disponibles en el instante de tiempo en que se
descargó el dataset, vía la API oficial de Spotify.

En la
\href{https://www.kaggle.com/datasets/yamaerenay/spotify-dataset-19212020-600k-tracks?resource=download}{página}
de donde descargamos la data, se menciona que las canciones extraídas
fueron publicadas entre los años 1921 y 2020.

Sin embargo, tras analizar la data descargada, notamos que hay canciones
cuyo año de lanzamiento figura entre 1900 y 2021. Incluso, una canción
del dataset figura que fue creada en el año 1900, pese a que, tras una
búsqueda web, hallamos que la banda que creó tal canción aún sigue
activa.

Por ello, es posible que, eventualmente, se requiera que nosotros mismos
empleemos el API oficial de Spotify para obtener la data de las
canciones cuyo identificador ya poseemos, gracias al dataset descargado.
Esto último con el fin de crear un nuevo dataset, ahora sin errores, con
el cual trabajaríamos como base datos de este proyecto.

También podríamos descartar a las filas donde encontremos errores como
el mencionado previamente respecto a la fecha de publicación de la
canción. Sin embargo, por ahora, seguiremos usando el dataset
descargado.

Entonces, el \textbf{universo} lo consideramos como el conjunto de
canciones disponibles en Spotify, \textbf{actualmente}.

En ese sentido, la \textbf{muestra} consiste del conjunto de canciones
de Spotify de las cuales tenemos información en las filas del dataset
descargado. Por ahora, es una muestra arbitraria (no se sabe como se
realizó el muestreo), pero, más adelante, nosotros mismos realizaremos
el muestreo, vía el API oficial de Spotify.

\hypertarget{descripciuxf3n-y-entendimiento-de-variables}{%
\subsubsection{Descripción y entendimiento de
variables}\label{descripciuxf3n-y-entendimiento-de-variables}}

Esta sección del capítulo dos la hemos desarrollado en el
\textbf{Jupyter notebook} presentado para este informe, así que no
incluiremos aquella descripción en este archivo.

El resto de este capítulo lo hemos incluido en un dashboard interactivo,
al cual puede acceder vía este
\href{https://lucio-cornejo.shinyapps.io/chapter-II-dashboard-INF03/}{link}.

\hypertarget{capuxedtulo-iii-preprocesamiento-de-datos}{%
\subsection{CAPÍTULO III: PREPROCESAMIENTO DE
DATOS}\label{capuxedtulo-iii-preprocesamiento-de-datos}}

En ese capitulo se detalla el proceso de preprocesamiento de los datos.
Por otro lado, la variable popularity es la variable dependiente del
estudio por lo cual esta variable no debe pasar por ningún tratamiento
de outliers, vacíos y/o transformación

\hypertarget{selecciuxf3n-de-registros-y-atributos}{%
\subsubsection{Selección de registros y
atributos}\label{selecciuxf3n-de-registros-y-atributos}}

Excluimos la variables ``artist'', pues esta no aporta información para
la predicción de popularidad de canciones.

\hypertarget{tratamiento-de-datos-atuxedpicos}{%
\subsubsection{Tratamiento de datos
atípicos}\label{tratamiento-de-datos-atuxedpicos}}

Si bien se ahorraría tiempo realizar el análisis de valores outliers de
las variables numéricas de forma automática donde se usaría el límite
superior e inferior teórico, para poder realizar un análisis minucioso,
las variables numéricas se analizarán de forma particular. El proceso
que seguimos fue cambiar el coeficiente que acompaña al rango
intercuartil al momento de definir los limites superiores e inferiores.

\textbf{Loudness}

Describe la sonoridad general de una pista en decibeles. Los valores de
sonoridad se promedian en toda la pista y son útiles para comparar la
sonoridad relativa de las pistas. La sonoridad es la cualidad de un
sonido que es el principal correlato psicológico de la fuerza física
(amplitud). Los valores suelen oscilar entre -60 y 0 db. Se observa que
la distribución de la variable se sesga hacia el lado derecho, en este
caso usamos a 1.7 como coeficiente. El porcentaje de valores atípicos es
9\% del total de los datos de esta columna. Por lo que es posible
generar imputaciones.

\textbf{Instrumentalness}

Predice si una pista no contiene voces. Los sonidos ``Ooh'' y ``aah'' se
consideran instrumentales en este contexto. Las pistas de rap o de
palabras habladas son claramente ``vocales''. Cuanto más se acerque el
valor de instrumentalización a 1,0, mayor será la probabilidad de que la
pista no tenga contenido vocal. Los valores superiores a 0,5 representan
pistas instrumentales, pero la confianza es mayor a medida que el valor
se acerca a 1,0. Se observa que la mayoría de los datos están
concentrado alrededor de cero, es decir la distribución esta sesgada a
la izquierda, en este caso usamos a 3 como coeficiente. El porcentaje de
valores atípicos es mayor a 20\% del total de los datos de esta columna.
Por lo que es no posible generar imputaciones.

\textbf{Liveness}

Detecta la presencia de una audiencia en la grabación. Valores más altos
de liveness representan una probabilidad incrementada de que la pista
haya sido realizada en vivo. Un valor de 0.8 provee una probabilidad
fuerte de que la pista sea en vivo. Se observa que los datos están
concentrados alrededor de 0.1 aproximadamente; no obstante tales datos
muestran un fuerte sesgo hacia la derecha, e incluso presentan un
ligerada bimodalidad. Usamos un coeficiente de 1.3 para determinar a los
valores outliers. El porcentaje de valores atípicos es de 9.51\%.

\textbf{Valence}

Es medida desde 0.0 hasta 1.0 que describe la positividad musical
expresada por la pista. Las pistas con un sonido de alta valence suenan
más positivos (e.g., felices, alegres, eufóricos), mientras que las
pistas con una baja valence suenan más negativas (e.g., tristes,
deprimentes, enojadas). Se observa que al distribución se asemeja a uns
distribución unirfore , donde parace que cada valor tomado por esta
variable se acerca a un mismo nivel de densidad. Usamos un coeficiente
de 1.4 para determinar a los valores outliers y el porcentaje de valores
atípicos es de 9\%.

\textbf{Tempo}

Es el tempo general estimado de una pista en beats por minutos (BPM, por
sus siglas en inglés). En terminología musical, el tempo es la velocidad
o ritmo de una pieza dada y se deriva directamente de una duración
promedio de beat. Se observa que los datos presentan múltiples modas,
aproximadamente alrededor de los valores 75, 100, 140, 175. Estos datos
presentan un ligero sesgo hacia la derecha. El porcentaje de valores
atípicos para esta variable es 8.90\%.

\textbf{Duration\_ms}

Es la duración de una pieza en milisegundos. Respecto a la distribución
de esta variable, se observa que la distribución de los datos es en
principio aparentemente degenerada (los datos están virtual o totalmente
reunidos en un punto). No obstante, tal aparente degeneración se explica
por el hecho de que existe una serie de piezas cuya duración puede ser
extremadamente larga. En este caso, si bien los valores atípicos
representan solo el 8.99\% del total de la muestra de valores para esta
variable, tal cantidad de valores atípicos, dados sus valores
extremadamente altos, tienen un efecto visual altamente significativo
sobre la interpretabilidad de la distribución de los datos.

\textbf{Danceability}

La danceability describe qué tan adecuada es una pista para bailar,
basada en una combinación de elementos musicales incluyendo el tempo, la
estabilidad rítmica (rhythm stability), la fuerza del beat (beat
strength), y una regularidad general. Un valor de 0.0 es menos danceable
1.0 es máximamente danceable. Se observa que los datos se concentran
alrededor de 0.65, y presentan un ligero sesgo hacia la izquierda. Los
valores atípicos representan 8.216\% de la muestra,para calcularlos se
usó el coeficiente 1.6 que acompaña al rango intercuartil.

\textbf{Energy}

La energía es una medida desde 0.0 a 1.0 y representa una medida
perceptiva de intensidad y actividad. Típicamente, las pistas
energéticas se sienten rápidas, de volumen alto (loud), y ruidosas. Por
ejemplo, el death metal tiene una alta energía, mientras que el preludio
de Bach da un puntaje bajo en la escala. Características perceptivas
contribuyen a este atributo incluyen rango dinámico, el volumen
percibió, el timbre, el onset rate, y la entropía general. Se observa
que los datos son ligeramente multimodales. con los datos ligeramente
más concentrados alrededor de 0.35, 0.55, 7.00 y 0.85. Los valores
atípicos para esta variable representan 9.32\% de la muestar y para
calcularlos se usó el coeficiente 1.4 que acompaña al rango
intercuartil.

\textbf{Speechiness}

Esta variableetecta la presencia de palabras habladas en una pieza.
Mientras más contenido hablado presenta una grabación (e.g., talk show,
audio book, poesía), más cercano a 1.0 será el valor del atributo. Los
valores superiores a 0.66 describen piezas que probablemente estén
hechas enteramente de palabras habladas. Valores entre 0.33 y 0.66
describen piezas que podrías contener tanto música como una parte oral,
ya sea en secciones o en capas, incluyendo casos como el rap. Valores
menores a 0.33 con mayor probabilidad representan múscia y otras piezas
non-speech-like. De manera similar al caso de la variable Duration\_ms,
aunque en medida mucho menor, esta variable, Speechiness, presenta en
principio una aparente distribución degenerada, concentrada alrededor 0.
No obstante, tal aparente naturaleza se debe al 10.39\% de valores
atípicos que presenta la muestra de datos para esta variable, cuyos
niveles están concentrados alrededor de 1. En otra palabras,
aproximadamente el 88\% de los datos se concentra alrededor de 0.0,
mientras que el 10.39\% lo hace alrededor de 1.0.

\textbf{Acousticness}

Una medida de confianza desde 0.0 hasta 1.0 sobre si la pieza es
acústica. 1.0 representa alta confianza en que la pieza es acústica
(\textgreater=0 \textbar{} \textless= 1). Se observa que los datos se
concentran en su mayoría alrededor de ambos valores extremos. La
distribución de los valores para esta variable es en tal sentido
bimodal. Los valores atípicos representan un 8.76\% del total de valores
de la muestra para esta variable.

\hypertarget{tratamiento-de-datos-vacuxedos}{%
\subsubsection{Tratamiento de datos
vacíos}\label{tratamiento-de-datos-vacuxedos}}

Los datos que disponemos no tienen valores vacíos, por lo que obviamos
este análisis.

\hypertarget{creaciuxf3n-y-transformaciuxf3n-de-variables}{%
\subsubsection{Creación y transformación de
variables}\label{creaciuxf3n-y-transformaciuxf3n-de-variables}}

Inicialmente la base de datos con la que contábamos tenía una reducida
cantidad de columnas, de las cuales algunas no aportaban con información
relevante para el objetivo del negocio. En ese sentido, transformamos
las variables name, time\_signature y realice\_data. De las cuales
obtenemos información que podrán servir para la predicción.

\textbf{Name}

Contiene el nombre de la canción, a partir de esta se generan dos
variables:

\begin{itemize}
\tightlist
\item
  Name\_lenght : contabiliza al cantidad de caracteres string del nombre
  de la canción omitiendo los espacios vacíos entre palabra y palabra.
\item
  Words\_name: contabiliza la cantidad de palabras que están presentes
  en el nombre de una canción.
\end{itemize}

Es importante recordar, que luego de generar ambas variables, se tiene
que eliminar la variable ``name'' para no caer en el problema de
multicolinealidad.

\textbf{Realise date}

Indica la fecha del lanzamiento de la canción incluyendo el año, mes y
día. Apartir de esat variable se crearon cuatro variables:

\begin{itemize}
\tightlist
\item
  Release\_year: año en el que se publicó la canción
\item
  Release\_month: mes en el que se publicó la canción
\item
  Release\_days: día en el que se publicó la canción
\item
  Release\_trim: trimestre en el que se publicó la canción
\end{itemize}

Posteriormente, eliminamos la variable realise date.

\textbf{Time\_signature}

Esta variable contiene información sobre el compás, este variable toma
valores de 3 a 7, asi discretizamos esta variable para que tome el valor
de 0 si los valores son mayores iguales a 0 y menores que 4; y tome el
valor de 1si los valores toman valores mayores iguales a 4.

\hypertarget{descripciuxf3n-de-variables-listas-para-el-modelamiento}{%
\subsubsection{Descripción de variables listas para el
modelamiento}\label{descripciuxf3n-de-variables-listas-para-el-modelamiento}}

\hypertarget{anuxe1lisis-de-correlaciuxf3n}{%
\subsubsection{Análisis de
correlación}\label{anuxe1lisis-de-correlaciuxf3n}}

Se observa que los colores que adoptan las casillas de la matriz de
correlación son rosados, lo cual indica que las variables tienen un
coeficiente de correlación que se encuentra entre 0.2 y 0.4, es decir el
nivel de correlación entre variables numéricas predictoras no es muy
alta.

\end{document}
